\documentclass{article}
\usepackage{graphicx}
\usepackage[margin=1in]{geometry} % Adjust margins as you like
\usepackage{amsmath}    % for advanced math
\usepackage{amssymb}    % for additional math symbols
\usepackage{setspace}
\usepackage{cite} 
\onehalfspacing


\begin{document}
\thispagestyle{empty}
\setlength{\parindent}{0pt}
%------------------------------------------
% Title row: left logo, center text, right logo
%------------------------------------------
\noindent
\includegraphics[width=2cm]{Picture1.png}%
\hfill
\begin{minipage}[c]{0.75\textwidth}
  \centering
  {\bfseries\LARGE Ben Gurion University of the Negev}\\
\vspace{0.3cm}  
{\bfseries\Large Faculty of Engineering Sciences}\\
\vspace{0.3cm}
{\bfseries\large Department of Mechanical Engineering}
\end{minipage}
\hfill
\includegraphics[width=2cm]{Picture2.png}

\vspace{2cm}

\begin{center}
    { \bfseries\large Research proposal}\\
\end{center}

\vspace{3cm}

\begin{center}
    { \bfseries\huge Turbulent Swirl Flow and Vortex Breakdown}\\
\vspace{1cm}
{\large Name: Benny Vradman}\\
\vspace{1cm}
{\large Date: 8.1.2025}
\end{center}
%------------------------------------------
\newpage
\pagenumbering{arabic}
\setcounter{page}{1}

\section{Introduction}
%whats swirl flow, how is it connected to turbulence.  what the project is about: 
%investigating turbulence in combustion as an example of the effect of swirl-turbulence connection.
The project investigates turbulence in swirl flow. Swirl flow is characterized by a helix flow swirl number \cite{altmann_caring_2019}.

Columnar vortices.

%significance of the research suggested:

\section{Project Objectives}
Review of the literature on the subject of measuring swirl flow and the theory of vortex breakdown. 


\bibliographystyle{plain} 
\bibliography{MyLibrary} 

\end{document}
