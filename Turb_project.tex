\documentclass{article}
\usepackage{graphicx}
\usepackage[margin=1in]{geometry} % Adjust margins as you like
\usepackage{amsmath}    % for advanced math
\usepackage{amssymb}    % for additional math symbols
\usepackage{setspace}
\usepackage{cite}
\usepackage{enumitem}
\onehalfspacing


\begin{document}
\thispagestyle{empty}
\setlength{\parindent}{0pt}

\noindent
\includegraphics[width=2cm]{Picture1.png}%
\hfill
\begin{minipage}[c]{0.75\textwidth}
  \centering
  {\bfseries\LARGE Ben Gurion University of the Negev}\\
\vspace{0.3cm}  
{\bfseries\Large Faculty of Engineering Sciences}\\
\vspace{0.3cm}
{\bfseries\large Department of Mechanical Engineering}
\end{minipage}
\hfill
\includegraphics[width=2cm]{Picture2.png}

\vspace{2cm}

\begin{center}
    { \bfseries\large Research project}\\
\end{center}

\vspace{3cm}

\begin{center}
    { \bfseries\huge Turbulent Flow course Project}\\
\vspace{1cm}
{\large Name: Benny Vradman}\\
\vspace{1cm}
{\large Date: 8.1.2025}
\end{center}
%------------------------------------------
\newpage
\pagenumbering{arabic}
\setcounter{page}{1}

\section{Introduction}
%whats swirl flow, how is it connected to turbulence.  what the project is about: 
%investigating turbulance in combustion as an example of the effect of swirl-turbulance connection.
The project investigates turbulence in swirl flow. Swirl flow is charactarized by a helix flow swirl number 

\section{Governing equations}

\subsection{dimensional analysis}
Swirl number:
\begin{equation}
    S(x)=\frac{1}{R(x)}\frac{G_\theta(x)}{G_x(x)}
\end{equation}
where $G_\theta$ is the time average of the axial component of angular momentum of the flow, 
$R$ is a characteristic radius of the swirling flow and %how calculated?
$G_x$ is the time average of the flow  rate of axial momentum \cite{vignat_suitability_2022}.
The meaning of this swirl number is a ration between the angular momentum and the axial momentum
 of the flow, divided by the radius of the flow, at any position $x$ along the cross section of the flow.\\

 connection the swirl number to Rossby number is:
\begin{align}
&Ro=\frac{U^*}{\Omega^*R} ,\\
&G_\theta=\pi\rho U^*\Omega R^4,\\
&G_x=\pi\rho U^{*2}R^2
\end{align}
\bibliographystyle{plain}
\bibliography{MyLibrary} 



\end{document}
